% run with 'xelatex --shell-escape'
\documentclass{beamer}

\usepackage[utf8]{inputenc}     % Umlaute
\usepackage[ngerman]{babel}     % Umlaute
\usepackage{minted}             % nice Code presentation (e.g. syntax highlighting)
\usepackage{svg}                % including svg
\usepackage{graphicx}           % including png, jpg
\usepackage{fontspec}           % use any of your installed fonts
\usepackage{tikz}               % here: to include a logo on every page

% define a color for the Stud.IP-Look
\definecolor{studipblue}{RGB}{48,84,136}

% Main theme
% There are many different themes available for Beamer. A comprehensive
% list with examples is given here:
% http://deic.uab.es/~iblanes/beamer_gallery/index_by_theme.html
\usetheme{Madrid}

% inner color theme (blocks etc.)
\usecolortheme{rose}
% Equations should look different
\usefonttheme[onlymath]{serif}

% define custom fonts
\usefonttheme{professionalfonts}
\usefonttheme{serif}
\setbeamerfont{title}{size=\Huge, family=\fontspec{Lato-Thin}}
\setbeamerfont{structure}{family=\fontspec{Lato-Thin}}
\setbeamerfont{title page}{family=\fontspec{Lato-Thin}}
\setbeamerfont{footline}{family=\fontspec{Lato-Light}}
\setbeamercolor{structure}{fg=studipblue}

% remove the navigation symbols in the lower right corner
\beamertemplatenavigationsymbolsempty

% Main title
\title{Make the Wall great again}
% A subtitle is optional and this may be deleted
\subtitle{\includegraphics[height=8mm]{../img/studip-logo@2x}}

\author{John Snow}
% Institute and University (use the '[...]' brackets to display a
% different/shorter version in the footline)
\institute[Night's Watch]{\inst{} Castle Black\\ The Wall}
% Conference name and date
\date{Choosing, \today}

% If there are more than one speaker and they are from different Institutes:
%\author{F.~Author\inst{1} \and S.~Another\inst{2}}
%% - Give the names in the same order as the appear in the paper.
%% - Use the \inst{?} command only if the authors have different
%%   affiliation.
%
%\institute[Universities of Somewhere and Elsewhere] % (optional, but mostly needed)
%{
%  \inst{1}%
%  Department of Computer Science\\
%  University of Somewhere
%  \and
%  \inst{2}%
%  Department of Theoretical Philosophy\\
%  University of Elsewhere}


% This is only inserted into the PDF information catalog. Can be left out.
\subject{Theoretical Computer Science}

% If you have a file called "university-logo-filename.xxx", where xxx
% is a graphic format that can be processed by latex or pdflatex,
% resp., then you can add a logo as follows:

% \pgfdeclareimage[height=0.5cm]{university-logo}{university-logo-filename}
% \logo{\pgfuseimage{university-logo}}

% Delete this, if you do not want the table of contents to pop up at
% the beginning of each subsection:
\AtBeginSubsection[]{\begin{frame}<beamer>{Outline}
                        \tableofcontents[currentsection,currentsubsection]
                     \end{frame}
}

% Start of the presentation
\begin{document}

% The 'frame'-command must not be indented!
\begin{frame}
    \titlepage
\end{frame}

% Set the logo in the top right corner of every frametitle
\addtobeamertemplate{frametitle}{}{%
\begin{tikzpicture}[remember picture,overlay]
\node[anchor=north east,yshift=2pt] at (current page.north east) {\includegraphics[height=0.6cm]{../img/studip-logo@2x}};
\end{tikzpicture}}

\begin{frame}{Outline}
    \tableofcontents
    % You might wish to add the option [pausesections]
\end{frame}

% Section and subsections will appear in the presentation overview
% and table of contents.
\section{First Main Section}

\subsection{First Subsection}

\begin{frame}[fragile]{First Slide Title}{Optional Subtitle}
    \begin{itemize}
        \item {My first point.}
        \item {My second point.}

        % Example for a code snippet inside an \item (without a leading bullet)
        \item[]{\begin{minted}[linenos, autogobble, numbersep=5pt, frame=lines, framesep=2mm]{java}
                    String title = "This is a unicorn in the sky";
                    // This is a comment
                    final double PI = 3.1415926535;
                \end{minted}
        }
    \end{itemize}
\end{frame}

\subsection{Second Subsection}

% You can reveal the parts of a slide one at a time
% with the \pause command:
\begin{frame}{Second Slide Title}
    \begin{itemize}
        \item {First item.
            \pause % The slide will pause after showing the first item
        }
        \item {Second item. $Inc_{0} = M * 42$}
  % You can also specify when the content should appear
  % by using <n->:
  \item<3-> {Third item.}
  \item<4-> {Fourth item.}
  % or you can use the \uncover command to reveal general
  % content (not just \items):
  \item<5-> {Fifth item. \uncover<6->{Extra text in the fifth item.}}
  \end{itemize}
\end{frame}

\section{Second Main Section}

\subsection{Another Subsection}

\begin{frame}{Blocks}
\begin{block}{Block Title}
You can also highlight sections of your presentation in a block, with it's own title
\end{block}
\begin{theorem}
There are separate environments for theorems, examples, definitions and proofs.
\end{theorem}
\begin{example}
Here is an example of an example block.
\end{example}
\end{frame}

% Placing a * after \section means it will not show in the
% outline or table of contents.
\section*{Summary}

\begin{frame}{Summary}
  \begin{itemize}
  \item
    The \alert{first main message} of your talk in one or two lines.
  \item
    The \alert{second main message} of your talk in one or two lines.
  \item
    Perhaps a \alert{third message}, but not more than that.
  \end{itemize}

  \begin{itemize}
  \item
    Outlook
    \begin{itemize}
    \item
      Something you haven't solved.
    \item
      Something else you haven't solved.
    \end{itemize}
  \end{itemize}
\end{frame}



% All of the following is optional and typically not needed.
\appendix
\section<presentation>*{\appendixname}
\subsection<presentation>*{For Further Reading}

\begin{frame}[allowframebreaks]
  \frametitle<presentation>{For Further Reading}

  \begin{thebibliography}{10}

  \beamertemplatebookbibitems
  % Start with overview books.

  \bibitem{Author1990}
    A.~Author.
    \newblock {\em Handbook of Everything}.
    \newblock Some Press, 1990.


  \beamertemplatearticlebibitems
  % Followed by interesting articles. Keep the list short. 

  \bibitem{Someone2000}
    S.~Someone.
    \newblock On this and that.
    \newblock {\em Journal of This and That}, 2(1):50--100,
    2000.
  \end{thebibliography}
\end{frame}

\end{document}

\grid
