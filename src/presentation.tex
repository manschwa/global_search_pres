% run with 'xelatex --shell-escape'
\documentclass{beamer}

\usepackage[utf8]{inputenc}     % Umlaute
\usepackage[ngerman]{babel}     % Umlaute
\usepackage{minted}             % nice Code presentation (Syntax Highlighting)
\usepackage{svg}                % including svg
\usepackage{graphicx}           % including pictures
\usepackage{fontspec}           % use any of your installed fonts
\usepackage{tikz}               % Logo in the top right corner

% define a color for the Stud.IP-Look
\definecolor{studipblue}{RGB}{48,84,136}

% Main theme
\usetheme{Madrid}

% Main color theme
\setbeamercolor{structure}{fg=studipblue}
% inner color themes (blocks etc.)
\usecolortheme{rose}
% Equations should look different
\usefonttheme[onlymath]{serif}

% define custom fonts
\usefonttheme{professionalfonts}
\usefonttheme{serif}
\setbeamerfont{title}{size=\Huge, family=\fontspec{Lato-Thin}}
\setbeamerfont{structure}{family=\fontspec{Lato-Thin}}
\setbeamerfont{title page}{family=\fontspec{Lato-Thin}}
\setbeamerfont{subsection in toc}{size=\small, family=\fontspec{Lato-Regular}}
\setbeamerfont{bibliography entry author}{family=\fontspec{Lato-Regular}}
\setbeamerfont{bibliography entry title}{family=\fontspec{Lato-Thin}}
\setbeamerfont{block title}{family=\fontspec{Lato-Thin}}
\setbeamerfont{block body}{size=\small, family=\fontspec{Lato-Regular}}
\setbeamerfont{block title example}{family=\fontspec{Lato-Thin}}
\setbeamerfont{block body example}{size=\small, family=\fontspec{Lato-Regular}}
\setbeamerfont{footline}{family=\fontspec{Lato-Light}}

%\setbeamertemplate{section in toc}{\inserttocsectionnumber.\,~\inserttocsection}
%\setbeamertemplate{subsection in toc}[default]%{\inserttocsectionnumber.\,~\inserttocsection}
\setbeamertemplate{sections/subsections in toc}[default]%{\inserttocsectionnumber.\,~\inserttocsection}
\setbeamertemplate{itemize items}[default]
\setbeamertemplate{enumerate items}[default]
\beamertemplatenavigationsymbolsempty

\title{Globale Suche}
% A subtitle is optional and this may be deleted
\subtitle{\includegraphics[height=8mm]{../img/studip-logo@2x}}

\author{Manuel Schwarz}
\institute[Uni Osnabrück]{\inst{} virtUOS\\ Universität Osnabrück}
\date{IWB, 31. Mai 2016}


% This is only inserted into the PDF information catalog. Can be left out.
%\subject{Theoretical Computer Science}

% If you have a file called "university-logo-filename.xxx", where xxx
% is a graphic format that can be processed by latex or pdflatex,
% resp., then you can add a logo as follows:

% \pgfdeclareimage[height=0.5cm]{university-logo}{university-logo-filename}
% \logo{\pgfuseimage{university-logo}}

% Delete this, if you do not want the table of contents to pop up at
% the beginning of each subsection:
%\AtBeginSubsection[]
%{
%  \begin{frame}<beamer>{Outline}
%    \tableofcontents[currentsection,currentsubsection]
%  \end{frame}
%}

% Start of the presentation
\begin{document}

\begin{frame}
    \titlepage
\end{frame}

% Set the logo in the top right corner of every frametitle
\addtobeamertemplate{frametitle}{}{%
\begin{tikzpicture}[remember picture,overlay]
\node[anchor=north east,yshift=2pt] at (current page.north east) {\includegraphics[height=0.6cm]{../img/studip-logo@2x}};
\end{tikzpicture}}

\begin{frame}{Überblick}
    \tableofcontents
    % You might wish to add the option [pausesections]
\end{frame}


% Section and subsections will appear in the presentation overview
% and table of contents.
\section{Das Ziel}
\subsection{Wunschliste}

\begin{frame}{Wunschliste}{Das Ziel}
    \begin{itemize}
        \item {\textit{eine} zentrale Suche\pause}
        \item {Google-ähnliche Suchleiste\pause}
        \item {mit maximal 1 Klick erreichbar\pause}
        \item {Quicksearch-Leiste (von überall erreichbar)\pause}
        \item {ansprechende Präsentation der Ergebnisse (Sortierung)\pause}
        \item {Filtermöglichkeit der Ergebnisse}
    \end{itemize}
\end{frame}


\section{Der Weg}
\subsection{Bisherige Suche}

% You can reveal the parts of a slide one at a time
% with the \pause command:
\begin{frame}{Bisherige Suche}{Der Weg}
    \begin{itemize}
        \item {nach Bereichen aufgeteilte Suche\pause}
        \item[]{%
            \begin{itemize}
                \item{Veranstaltungen}
                \item{Archiv}
                \item{Personen}
                \item{Einrichtungen}
                \item{Ressourcen\pause}
            \end{itemize}
        }
    \item {Quicksearch-Leiste \textit{nur} für Veranstaltungen\pause}
    \item {Code nicht zentral an einer Stelle\pause}
    \item {Keine zusätzlichen Daten/Tabellen}
  \end{itemize}
\end{frame}


\subsection{Plugins}
\subsubsection{ItelligentSearch (fl0)}

\begin{frame}{IntelligentSearch}{Plugin (fl0)}
    \begin{itemize}
        \item {Zentrale Suche}
        \item[]{%
            \begin{itemize}
                \item{Veranstaltungen}
                \item{Personen}
                \item{Einrichtungen}
                \item{Ressourcen}
                \item{Vorlesungsverzeichnis}
                \item{Dokumente/Dateien}
                \item{Foreneinträge}
                \item{Wiki\pause}
            \end{itemize}
        }
    \item {Quicksearch-Leiste\pause}
    \item {zwei zusätzliche Tabellen}
    \begin{itemize}
        \item {search\_object}
        \item {search\_index}
    \end{itemize}
  \end{itemize}
\end{frame}


\subsubsection{GlobalSearchPlugin (Krassmus)}

\begin{frame}{GlobalSearchPlugin}{Plugin (Krassmus)}
    \begin{itemize}
        \item {Zentrale Suche}
        \item[]{%
            \begin{itemize}
                \item{Veranstaltungen}
                \item{Personen}
                \item{Dokumente}
                \item{Foreneinträge}
                \item{Ressourcen\pause}
            \end{itemize}
        }
        \item {keine Quicksearch-Leiste\pause}
        \item {eine zusätzliche Tabelle (searchindex)\pause}
        \item {nette Präsentation}
    \end{itemize}
\end{frame}


\begin{frame}{Probleme}
    \begin{itemize}
        \item {Erstellen der Index-Tabelle\pause}
            \begin{itemize}
                \item {bisher nur manuell mit root-Rechten}
                \item {keine Automatisierung\pause}
                \item {neue Datensätze nicht sofort auffindbar\pause}
                \item {wird immer komplett neu erstellt\pause}
            \end{itemize}
        \item {Rechtefrage (Wer darf was sehen/finden?)\pause}
            \begin{itemize}
                \item {Root und Admin sollten alles sehen dürfen\pause}
                \item {Dateien, Wiki, Forum (im Veranstaltungskontext)\pause}
                \item {Rest: global für jeden zu finden (visibility?)\pause}
            \end{itemize}
        \item {Aufbereitung der Ergebnisse\pause}
            \begin{itemize}
                \item {Sortierung}
                \item {Priorisierung\pause}
                \item {Gruppierung\pause}
                \item {Beispiel: Suche nach Dozent}
            \end{itemize}
    \end{itemize}
\end{frame}


\subsection{Idee}

\begin{frame}{Idee(n)}
    \begin{itemize}
        \item {keine Index-Tabelle\pause~\uncover{- kein Aktualisierungsproblem\pause}}
        \item {Brute Force?\pause}
        \item {Activities?!\pause}
            \begin{itemize}
                \item {extra Tabelle\pause}
                \item {wächst dynamisch\pause}
                \item {nicht alle Einträge relevant (nur created/deleted)\pause}
                \item {Activities gibt es nicht rückwirkend\pause}
                \begin{itemize}
                    \item {Erstellung mit dem Start der Installation\pause}
                    \item {einmaligen/einfachen Activity-Generator\pause}
                \end{itemize}
                \item {Suche:\pause}
                \begin{itemize}
                    \item {content-Spalte durchsuchen\pause}
                    \item {object\_type, object\_id, context\pause}
                \end{itemize}
            \end{itemize}
    \end{itemize}
\end{frame}

%\begin{frame}{Blocks}
%\begin{block}{Block Title}
%You can also highlight sections of your presentation in a block, with it's own title
%\end{block}
%\begin{theorem}
%There are separate environments for theorems, examples, definitions and proofs.
%\end{theorem}
%\begin{example}
%Here is an example of an example block.
%\end{example}
%\end{frame}

\end{document}
